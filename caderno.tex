\documentclass[a4paper]{article}
\usepackage{graphicx}
\usepackage{xcolor}
\usepackage{lineno}
\usepackage[brazil]{babel}
\usepackage[utf8]{inputenc}
\usepackage[T1]{fontenc}
\usepackage[autostyle]{csquotes}
\usepackage[colorinlistoftodos,portuguese,linecolor=black]{todonotes}
\usepackage{marginnote}
\usepackage{hyperref}
\usepackage[default,scale=0.95]{opensans}
\usepackage{fancyhdr}
\usepackage{pgf,tikz,pgfplots}
\usepackage{amsfonts}
\usepackage{amssymb}
\usepackage{mathtools}
\usepackage{mathrsfs}
\usepackage{nameref}
\usepackage{indentfirst}
\usepackage{array}
\usepackage{cancel}
\usepackage{enumitem}
\usepackage[alf]{abntex2cite}

\setlength{\parskip}{\baselineskip}
%\setlength{\parindent}{0pt}%

\usetikzlibrary{arrows,chains,fit,quotes,positioning,shapes}
\pgfplotsset{compat=1.13}

\newcommand{\aula}[2][]{\todo[inline,color=blue!10, #1]{\textbf{#2}}}
\newcommand{\nota}[2][]{\todo[color=yellow!30,linecolor=black, #1]{#2}}
\newcommand{\revisar}[2][]{\todo[color=red!30,linecolor=black, #1]{#2}}
\newcommand{\notagrande}[2][]{\todo[inline,color=orange!30, #1]{#2}}
\newcommand*\bolinha[1]{\tikz[baseline=(char.base)]{
		\node[shape=circle,draw,inner sep=2pt] (char) {#1};}}
\newcommand{\notaesq}[2][]{{%
		\let\marginpar\marginnote
		\reversemarginpar
		\renewcommand{\baselinestretch}{0.8}%
		\nota[#1]{#2}}}
\renewcommand\CancelColor{\color{red}}
\newcommand\defeq{\mathrel{\overset{\makebox[0pt]{\mbox{\normalfont\tiny\sffamily def}}}{=}}}
\newcommand\noteq{\mathrel{\overset{\makebox[0pt]{\mbox{\normalfont\tiny\sffamily not}}}{=}}}

\definecolor{ffqqqq}{rgb}{1.,0.,0.}
\definecolor{qqzzcc}{rgb}{0.,0.6,0.8}
\definecolor{eqeqeq}{rgb}{0.8784313725490196,0.8784313725490196,0.8784313725490196}

\fancypagestyle{plain}{
	\fancyhf{} %Clear Everything.
	\fancyfoot[C]{\thepage} %Page Number
	\renewcommand{\headrule}{\hrule height 2pt \vspace{1mm}\hrule height 1pt}
	\renewcommand{\footrulewidth}{1pt}
	\fancyfoot[L]{BOTTOM LEFT}
	\fancyfoot[R]{BOTTOM RIGHT}
	\fancyhead[LE]{TOP LEFT, EVEN PAGES}
	\fancyhead[RO]{TOP RIGHT, ODD PAGES}
}
\fancyfoot[C]{\sffamily\textbf{\thepage}}

\renewcommand\seriesdefault{l}
\renewcommand\mddefault{l}
\renewcommand\bfdefault{sb}

\hypersetup{
	colorlinks  = true,
	allcolors   = black,
	pdftitle    = {Caderno de IPE},
	pdfsubject  = {Estatística},
	pdfauthor   = {Caio César Carvalho Ortega},
	pdfcreator  = {Caio César Carvalho Ortega},
	pdfproducer = {Caio César Carvalho Ortega},
	pdfkeywords = {estatística, probabilidade}
}

\definecolor{Red}{rgb}{0.9,0.0,0.1}
%\definecolor{Blue}{rgb}{0.1,0.1,0.9}
%\hyphenation{}
\pagestyle{fancy}

% ----------------------------------------------------------

\begin{document}
	
	\section{Prólogo}
	
	As anotações e considerações que se seguem foram realizadas de maneira autônoma e não são fruto de orientação por parte da universidade e/ou de qualquer membro do corpo docente. Foram realizadas para fins de estudo, sendo, portanto, reflexo de um esforço de cunho pessoal para melhor apreensão do conteúdo discutido em sala.
	
%	\nocite{ross2010}
%	\nocite{morin2016}
%	\nocite{lipson2011}

	\aula{Início da aula de 25/06/2019}
	
	\section{Introdução à Estatística Descritiva}
	
	Conteúdo proposto:
	
	\begin{itemize}
		\item Conceitos básicos
		\begin{itemize}
			\item Elemento
			\item Dado e variável
			\item Valor e conjunto de valores
			\item Tipo de dado e tipo de variável
		\end{itemize}
		\item Tabelas
		\begin{itemize}
			\item Frequência
		\end{itemize}
		\item Gráficos
		\begin{itemize}
			\item Variável qualitativa
			\item Variável quantitativa
		\end{itemize}
	\end{itemize}
	
	A coisa mais importante na Estatística são os dados, designados como elementos, que podem ser muita coisa, uma entidade, um sujeito, uma pessoa, uma observação, um experimento. Por exemplo: o professor nasceu em Avaré, eis um dado; dizer que o céu desta noite está estrelado é uma observação, ou seja, um dado que veio de uma observação; já a ideia do experimento é ser um dado fruto de um processo controlado, como uma receita de bolo. Tudo isso são dados.
	
	O dado é um fato, uma característica, de um elemento. Nacionalidade, gentílico, cidade-natal, avaliação do bolo produzido com a receita (se ficou gostoso ou não).
	
	Valor é um agrupamento de dados, ou ainda, valores que os dados poderão ter. Podemos falar também em conjunto de valores.
	
	Quanto ao tipo de dado ou variável, estes podem ser qualitativos (também chamados de categorizados) e quantitativos. Qualitativa: todos aqueles dados e variáveis que não são números (exemplo: local de nascimento), subdivididos em nominal (exemplo: bolo gostoso ou bolo ruim) e ordinal (exemplo: altura de uma pessoa); uma variável qualitativa pode ser uma variável discreta (que só pode assumir valores inteiros) ou contínua (que pode assumir qualquer valor, podendo ser decimal).
	
	Tomando por base o exercício 2 do arquivo ``apostila completa.pdf'', Munhoz começa a responder cada uma das questões.
	
	\begin{enumerate}[label=\alph*.]
		\item 6
		\item 4
		\item 24
		\item Qualitativa: gênero\\ Quantitativa: idade, questão do aborto, classificação
		\item Não
	\end{enumerate}
	
	Feito o exercício, devemos compreender os conceitos de (i) frequência absoluta, que pode ser definido como: quantidade de ocorrência de um valor ou conjunto de valores; (ii) tabela de frequência absoluta, que pode ser definida como a tabela em que as linhas expressam as frequências de cada valor ou conjunto de valores.
	
	Munhoz agora parte para o exercício 8D no mesmo arquivo PDF. Professor inicia a construção de uma tabela de frequências absolutas. Em seguida, complementa com a coluna de frequências relativas. Vide \autoref{tab:tab1}.
	
	\begin{table}[]
		\centering
		\caption{Opinião dos funcionários sobre troca de gerente (G = gosta; N = não gosta)}
		\begin{tabular}{|l|l|l|}
			\hline
			\textbf{Opinião} & \textbf{Frequência absoluta} & \textbf{Frequência relativa} \\ \hline
			G & 9 & 45\% \\ \hline
			N & 11 & 55\% \\ \hline
		\end{tabular}
		\\ \vspace{1mm} Fonte: questão 8
		\label{tab:tab1}
	\end{table}
	
	Tabelas possuem título, corpo e fonte.
	
	Frequência relativa é o percentual do aparecimento de cada valor.
	
	Professor agora parte para o gráfico, dizendo que este torna a visualização mais fácil. Aponta que o gráfico pode ser de coluna ou de setores (também chamado de ``gráfico de pizza'').
	
	Desenha um gráfico de colunas e um gráfico de setores.
	
	Munhoz sugere a utilização do histograma para gráficos de variáveis quantitativas. O histograma também é conhecido como damo-e-folhas (vide \autoref{fig:damofolhas1}). Define-o como gráfico de colunas em faixas justapostas.
	
	\begin{figure}[]
		\centering
		\caption{Damo-e-folhas para a pontuação}
		\begin{tabular}{l|lllll}
			2 & 8 & 6 &  &  &  \\
			3 & 5 &  &  &  &  \\
			2 & 6 & 9 & 5 & 4 &  \\
			6 & 4 & 8 & 0 &  &  \\
			9 & 2 & 5 & 2 & 0 & 9
		\end{tabular}
		\\ \vspace{1mm} Fonte: questão 10E
		\label{fig:damofolhas1}
	\end{figure}
	
	Para permitir a construção do histograma, agrupa as pontuações em classes (vide \autoref{labeltab:pontuacaoalunos}).
	
	\begin{table}[]
		\centering
		\caption{Pontuação dos alunos}
		\begin{tabular}{l|l|l|l}
			\textbf{Pontuação} & \textbf{Freq. absoluta} & \textbf{Freq. acum.} & \textbf{Freq. acum. (percent.)} \\ \hline
			50-59 & 3 & 3 & 15\% \\
			60-59 & 2 & 5 & 25\% \\
			70-79 & 5 & 10 & 50\% \\
			80-89 & 4 & 14 & 70\% \\
			90-99 & 6 & 20 & 100\%
		\end{tabular}
		\\ \vspace{1mm} Fonte: questão 10E
		\label{tab:pontuacaoalunos}
	\end{table}
	
	\subsection{Teste no Tidia}
	
	Questões 2c, 4e, 7 frequência absoluta de A e 11 frequência acumulada percentual da sexta classe da segunda lista, cada uma correspondendo às questões 1 a 4 do teste na plataforma, respectivamente.
	
	\aula{Início da aula de 28/06/2019}
	
	\section{Medidas de posição e variabilidade}
	
	\subsection{Medida de posição média para dados brutos}
	
	$X_{1}, X_{2}, X_{3}$
	
	Média = $M = \dfrac{X_{1} + X_{2} + \dots + X_{n}}{n}$
	
	\subsubsection{Exemplo}
	
	Vide gastos na \autoref{tab:gastomes}.
	
	\begin{table}[]
		\centering
		\caption{Gasto por mês}
		\begin{tabular}{l|l}
			\textbf{Mês} & \textbf{Gasto} \\ \hline
			Janeiro & 200 \\
			Fevereiro & 500 \\
			Março & 600
		\end{tabular}
		\\ \vspace{1mm} Fonte: sem fonte
		\label{tab:gastomes}
	\end{table}
	
	O gasto médio então é expresso por:
	
	$M = \dfrac{200+500+600}{3} = \dfrac{1300}{3} \cong 433$
	
	\subsubsection{Notação}
	
	$M$ é a média populacional e $\bar{x}$ é a média de uma amostra; $f$ é a frequência absoluta.
	
	\subsection{Medida de posição média para dados agrupados}
	
	\notagrande{Passar a limpo da foto.}
	
	\subsection{Dados arupados por faixas de valores}
	
	Troca a faixa pelo ponto central (o ponto médio) e calcula a média. Veja o exemplo da \autoref{tab:agrup6e}
	
	\begin{table}[]
		\centering
		\caption{Pontuação dos alunos}
		\begin{tabular}{l|l|l|l|l}
			\textbf{Chave} & \textbf{X} & \textbf{fa} & \textbf{f} & \textbf{fX} \\ \hline
			2-6 & 4 & 2 & 20\% & 0,8 \\
			7-11 & 9 & 3 & 30\% & 2,7 \\
			12-16 & 14 & 4 & 40\% & 5,6 \\
			17-21 & 19 & 1 & 10\% & 1,9 \\ \hline
			- & - & 10 & 11
		\end{tabular}
		\\ \vspace{1mm} Fonte: questão 6E
		\label{tab:agrup6e}
	\end{table}
	
	Logo a média é $11$ ($M=11$).
	
	\section{Média da variabilidade}
	
	\subsection{Desvio quadrático}
	
	Desvio quadrático é a diferença do quadrado do valor:
	
	\begin{equation*}
		\left( X_{i} -M \right)^{2}
	\end{equation*}
	
	Considerando novamente os gastos na \autoref{tab:gastomes} e que $M = 430$, temos:
	
	Desvio quadrático do $500$.
	
	\notagrande{Não deu tempo de copiar}
	
	Calculemos para os gastos dos exemplos anteriores (\autoref{tab:gastomesxi}).
	
	\begin{table}[]
		\centering
		\caption{Gasto por mês}
		\begin{tabular}{l|l}
			\textbf{Valor} & \textbf{$(X_{i}-M)^{2}$} \\ \hline
			200 & 52900 \\
			500 & 4900 \\
			600 & 28900 \\ \hline
			- & 86700
		\end{tabular}
		\\ \vspace{1mm} Fonte: sem fonte
		\label{tab:gastomesxi}
	\end{table}
	
	\subsection{Variância}
	
	Variância é a média dos desvios quadráticos, expressa com o símbolo $\sigma^{2}$ (sigma ao quadrado). Para a variância dos dados $X{1}, X_{2}, \dots, X_{n}$ com média $M$:
	
	\begin{equation*}
		\sigma{2} = \dfrac{(X_{1} + M)^{2} + (X_{2}  + M)^{2} + \dots + (X_{n} - M)^{2} }{n}
	\end{equation*}
	
	Adotemos novamente os gastos dos exemplos anteriores para cálculo da variância:
	
	$\sigma^{2} = \dfrac{52900 + 4900 + 28900}{3} = \dfrac{86700}{3} = 28900$
	
	\subsection{Coeficiente de variação}
	
	É o $CV$. A notação é $CV = \dfrac{\sigma}{M}$. Exemplo:
	
	$CV = \dfrac{170}{430} = 39\%$
	
	\section{Variância para dados brutos}
	
	Sendo $f =$ frequência relativa, temos a \autoref{tab:vdb}.
	
	\begin{table}[]
		\centering
		\begin{tabular}{l|l|l|l|l}
			\textbf{$X$} & \textbf{$f$} & \textbf{$fX$} & \textbf{$(X_{i} - M)^{2}$} & \textbf{$f(X_{i}-M)^{2}$} \\ \hline
			$X_{1}$ & $f_{1}$ & $f_{1}X_{1}$ & $(X_{1} - M)^{2}$ & $f_{1}(X_{1}-M)^{2}$ \\
			$X_{1}$ & $f_{1}$ & $f_{1}X_{1}$ & $(X_{1} - M)^{2}$ & $f_{1}(X_{1}-M)^{2}$ \\
			$X_{1}$ & $f_{1}$ & $f_{1}X_{1}$ & $(X_{1} - M)^{2}$ & $f_{1}(X_{1}-M)^{2}$ \\ \hline
			- & - & $M$ & - & $\sigma{2}$
		\end{tabular}
		\\ \vspace{1mm} Fonte: N/A
		\label{tab:vdb}
	\end{table}
	
	\subsection{Variância populacional e desvio populacional}

	\notagrande{Passar a limpo da foto.}
	
	\subsection{Percentis}
	
	É uma medida de posição. Sendo o percentil um ponto entre $p\%$ e $(-1p)\%$, temos:
	
	\begin{equation*}
		p = \underbracket{p \times n}_{\text{arredondado pra cima}}
	\end{equation*}
	
	$n$ é a quantidade de dados.
	
	Exemplo a partir da questão 3A, alínea (d):
	
	$p_{75}$ está na posição $75\% \times 8 = 6$
	
	O ranking é expresso pelos dados ordenados como \bolinha{30} \bolinha{36} \bolinha{42} \bolinha{42} \bolinha{44}
	\bolinha{46} \bolinha{50} \bolinha{54}
	
	Então $p_{75} = \dfrac{46+50}{2} = 48$
	
	$p_{75} = 48 = Q_{3}$ (3o. quartil)
	
	$p_{25} = Q_{1}$ (1o. quartil)
	
	$p_{25} = \dfrac{36 + 42}{2} = 39$
	
	$p_{50} = Q_{2} =$ mediana  $=\dfrac{42 + 44}{2} = 43$
	
	\notagrande{Box-plot na foto. Passar a limpo}
	
	Exemplo respondendo a questão 9C:
	
	$CV_{UTC} = \dfrac{0,84}{2,8} = 30\%$
	
	$CV_{UTK} = \dfrac{0,84}{2,4} = 37\%$
	
	Logo, a UTK tem notas mais dispersas.
	
	\notagrande{A P1 ficou para dia 19/7!}
	
	Responder no Tidia:
	
	\begin{enumerate}
		\item 3 (a + c + d)
		\item 7 (a + b)
		\item 10 (cidade e dispersão)
		\item 17 resposta: 131
	\end{enumerate}

	\section{Probabilidade: contextos básicos}

	\begin{itemize}
		\item Ementa:
		\begin{itemize}
			\item Probabilidade
			\item Probabilidade da ação
			\item Probabilidade condicional
		\end{itemize}
	\end{itemize}

	\subsection{Definições}
	
	\textbf{Evento}: conjunto de resultados.
	
	\textbf{Espaço amostral}: conjunto de todos os resultados possíveis
	
	\textbf{Ponto amostral}: é um elemento do espaço amostral
	
	\subsection{Exemplo}
	
	\noindent Dado o lançamento de uma moeda, são:
	
	$A =$ espaço amostral
	
	$A = \{cara,coroa\}$
	
	Pontos amostrais: cara, coroa
	
	\subsection{Exemplo}
	
	\noindent Dado o lançamento de duas moedas, são:
	
	$A =$ espaço amostral
	
	$A = \{(cara,coroa),(coroa,cara),(cara,cara),(coroa,coroa)\}$
	
	Pontos amostrais: (cara,coroa), (coroa,cara), (cara,cara), (coroa,coroa)
	
	$E =$ é o evento, no caso, \textbf{sair uma cara somente}
	
	$E = \{(cara,coroa),(coroa,cara)\} \subset A$
	
	\subsection{Definição de probabilidade}
	
	$P(E) =$ probabilidade do evento $E$
	
	$E \subset A, A = $espaço amostral
	
	$P(E) = \dfrac{n(E)}{n(A)} = \dfrac{\text{quantidade de elementos de $E$}}{\text{quantidade de elementos de $A$}}$
	
	\subsection{Exemplo}
	
	Considere o seguinte experimento: lançamento de uma moeda. Temos:
	
	$A = \{cara,coroa\}$
	
	$E =$ sair cara
	
	$E = \{(cara)\}$
	
	$P(E) = P(cara) = \dfrac{n(E)}{n(A)} = \dfrac{1}{2} = 50\%$
	
	\subsection{Exemplo}
	
	Considere o seguinte experimento: lançamento de duas moedas. Temos:
	
	$E =$ sair só uma cara
	
	$P(E) = \dfrac{2}{4} = \dfrac{1}{2} = 50\%$
	
	\subsection{Definição frequentista de probabilidade}
	
	Nela, a probabilidade ($P$) é igual a frequência relativa. Por exemplo: lançar uma moeda e esperar $500$ caras.
	
	\subsection{Exemplo}
	
	\noindent Referência: 3A, apostila 4
	
	\begin{enumerate}[label=\alph*.]
		\item $\{0,1,2,3,4\}$ \\ $5$
		\item $0$, cujo $P = 60 \div 200 = 30\%$ \\
			$1$, cujo $P = 80 \div 200 = 40\%$ \\ 
			$2$, cujo $P = 40 \div 200 = 20\%$ \\
			$3$, cujo $P = 16 \div 200 = 8\%$ \\
			$4$, cujo $P = 4 \div 200 = 2\%$
		\item $P(V=0) = 30\%$
		\item $P(V \geq 2) = 20\% + 8\% + 2\% = 30\%$
		\item $P(V=1$ ou $V=2) = 40\% + 20\% = 60\%$ ou, alternativamente \\
			$P(V=1$ ou $V=2) = \dfrac{120}{200} = 60\%$
		\item $P(V=2$ ou $V=1$ ou $V=0) = \dfrac{180}{200} = 90\%$ ou, alternativamente \\
			$P(V2$ ou $V=1$ ou $V=0) = 20\% + 40\% + 30\% - 90\%$
	\end{enumerate}
	
	\subsection{Probabilidade da união}
	
	\subsubsection{Proposição}
	
	Proposição: $P(A \cup B) = P(A) + P(B) - P(A \cap B)$
	
	Prova: $n =$ quantidade ttoal de possibilidades $=$ tamanho do espaço amostral.
	
	Temos:
	
	$n_A\cup B = n_A + n_B - n_{A \cap B}$
	
	$\dfrac{n_{A\cup B}}{n} = \dfrac{n_{A}}{n} + \dfrac{n_B}{n} - \dfrac{n_{A \cap B}}{n}$

	Logo:
	
	$P(A \cup B) = P(A) + P(B) - P(A \cap B)$
	
	\subsubsection{Exemplo}
	
	Professor utiliza o exemplo $n_{B \cup C} = n_{B} + n_{C} - n_{B \cap C}$, vide \autoref{fig:exemplo2conj}.
	
	\begin{figure}[h]
	\centering
		\begin{tikzpicture}[
		thick]
		\draw [fill=cyan, fill opacity=0.5] (0,0) circle (2cm);
		\draw [fill=orange, fill opacity=0.5] (3,-1) circle (2.5cm);
		\draw (0,0) ++(120:2cm) -- ++(120:2.2cm) node [fill=white,inner sep=5pt](a){$B$};
		\draw (3, -1) ++(30:2.5cm) -- ++(30:2.6cm) node [fill=white,inner sep=5pt](b){$C$};
		\end{tikzpicture}
		\caption{Observar interseção}
		\label{fig:exemplo2conj}
	\end{figure}
	
	\subsubsection{Definição}
	
	$A$ e $B$ são eventos excludentes \hspace{5mm} $\leftrightarrow A \cap B = \emptyset$
	
	\subsubsection{Proposição}
	
	$A$ e $B$ são eventos excludentes \hspace{5mm} $\leftrightarrow P(A \cup B) = P(A) + P(B)$
	
	\subsubsection{Exemplo}
	
	Referência: exercício 1A da apostila 4
	
	$P(A) = 23\%$
	
	$P(A) = 19\%$
	
	$P(A$ ou $B) = P(A \cup B) = 38\%$
	
	\begin{enumerate}[label=\alph*.]
		\item $P(A \cap B) = ?$ \\
			\\
			Temos: \\
			\\
			$P(A \cup B) = P(A) + P(B) - P(A \cap B)$ \\
			$P(A \cap) B) = P(A) + P(B) - P(A \cup B)$ \\
			$P(A \cap) B) = 23\% + 19\% - 36\% = 42\% - 38\%$ \\
			$P(A \cap) B) =	4\%$
		\item Não são eventos excludentes, pois $A \cap B \neq \emptyset$ devido a $P(A \cap B) > 0$
	\end{enumerate}

	\subsection{Probabilidade condicional}
	
	\subsubsection{Definição}
	
	$P(A | B) = \dfrac{N_{A \cap B}}{N_B}$
	
	\subsubsection{Exemplo}
	
	Considere a retirada de bolas pretas e brancas numa caixa:
	
	\bolinha{P} \bolinha{P} \bolinha{P} \bolinha{B} \bolinha{B}
	
	$P(\text{sair bola preta na 2a. vez }|\text{ sair bola branca na 1a. vez}) = \dfrac{3}{4} = 75\%$

	$P(\text{sair bola preta na 2a. vez }|\text{ sair bola branca na 1a. vez}) = \dfrac{2}{4} = 50\%$	
	
	Experimento 1: tira uma bola
	
	Experimento 2: tira uma bola
	
	\bolinha{P} \bolinha{P} \bolinha{P} \bolinha{B} \bolinha{B} $\rightarrow$ 1a. vez $\rightarrow$ \bolinha{P} \bolinha{P} \bolinha{B} \bolinha{B} $\rightarrow$ 2a. vez

	\subsubsection{Exemplo}
	
	$A = \{1,2,3,4,5\}$
	
	$B = \{1,2\}$
	
	$C = \{7,8,9\}$
	
	\subsubsection{Proposição}
	
	$P(A|B) = \dfrac{P(A \cap B)}{P(B)}$
	
	\subsubsection{Prova}
	
	$P(A|B) = \dfrac{n_{A \cap B}}{n_B}$
	
	$P(A|B) = \dfrac{\frac{n_{A \cap B}}{n}}{\frac{n_B}{n}} = \dfrac{P(A \cap B)}{P(B)}$
	
	\subsubsection{Consequência}
	
	$P(A \cap B) = P(A|B) \cdot P(B)$
	
	\subsubsection{Definição}
	
	Eventos $A$ e $B$ são independentes \hspace{5mm} $\leftrightarrow P(A|B) = P(A)$
	
	\subsubsection{Proposição}
	
	$A$ e $B$ são eventos independentes \hspace{5mm} $\leftrightarrow P(A \cap B) = P(A) \cdot P(B)$
	
	\subsubsection{Prova}
	
	$P(A \cap B) = P(A|B)P(B) = P(A)P(B)$
	
	\subsubsection{Exemplo}
	
	Referência: exercício 8B da apostila 4
	
	\begin{enumerate}[label=\alph*.]
		\item $P(\text{cliente mulher}) = \dfrac{30 + 50}{200} = \dfrac{80}{200} = 40\%$
		\item $P(\text{cliente homem}) = \dfrac{100}{200} = 50\%$
		\item $P(\text{solteira }|\text{ mulher}) = \dfrac{30}{80} = 37,5\%$
		\item $\dfrac{12\cancel{0}}{20\cancel{0}} = 60\%$
		\item $P(\text{casado }|\text{ homem}) = \dfrac{100}{120} = 83,3\%$
		\item Não são, pois sõa iguais
		\item $P(\text{solteira }|\text{ mulher}) = \dfrac{30}{80} = 37,5\%$ \\
			$P(\text{solteiro}) = \dfrac{50}{200} = 25\%$\\
	\end{enumerate}

	\subsection{Teste no Tidia}
	
	Relação entre questão do Tidia e questão da apostila 4:
	
	\begin{enumerate}
		\item 7C
		\item 9D
		\item 11C
		\item 10, considerando: \\
			$x+y$ \\
			$x = P(A \cap B) - P(A) \cdot P(B)$ \\
			$y = P(A \cup B) - P(A) - P(B)$
	\end{enumerate}
	
	 ----------------------------------------------------------
	
	
	%
	
	\listoftodos[Lista de anotações]
	\addcontentsline{toc}{section}{Lista de anotações}
		
	\bibliography{fontes.bib}
	\addcontentsline{toc}{section}{Referências}

\end{document}


